\documentclass{article}

\usepackage{fancyhdr}
\usepackage{extramarks}
\usepackage{amsmath}
\usepackage{amsthm}
\usepackage{amsfonts}
\usepackage{tikz}
\usepackage[plain]{algorithm}
\usepackage{algpseudocode}

\usetikzlibrary{automata,positioning}

\usepackage{graphicx}
\graphicspath{ {./images/} }

%
% Basic Document Settings
%

\topmargin=-0.45in
\evensidemargin=0in
\oddsidemargin=0in
\textwidth=6.5in
\textheight=9.0in
\headsep=0.25in

\linespread{1.1}

\pagestyle{fancy}
\lhead{Yousef Alaa Awad}
\chead{\hmwkClass\: \hmwkTitle}
\rhead{\firstxmark}
\lfoot{\lastxmark}
\cfoot{\thepage}

\renewcommand\headrulewidth{0.4pt}
\renewcommand\footrulewidth{0.4pt}

\setlength\parindent{0pt}

%
% Create Problem Sections
%

\setcounter{secnumdepth}{0}
\newcounter{partCounter}
\newcounter{homeworkProblemCounter}
\setcounter{homeworkProblemCounter}{1}

\newcommand{\hmwkTitle}{Homework\ \#7}
\newcommand{\hmwkClass}{Power Systems Economics}

%
% Title Page
%

\title{
    \vspace{2in}
    \textmd{\textbf{\hmwkClass:\ \hmwkTitle}}\\
    \normalsize\vspace{0.1in}
    \vspace{3in}
}

\author{Yousef Alaa Awad}

% Problems start here
\begin{document}

\maketitle
\pagebreak

\section{4.10}
\textbf{Given:} Consider a market for electrical energy that is supplied by two generating companies whose cost functions are:
$$ C_A = 36*P_A\frac{\$}{h} $$
$$ C_B = 31*P_B\frac{\$}{h} $$
The inverse demand curve for this market is estimated to be:
$$ \pi = 120 - D\frac{\$}{\text{MWh}} $$
Assuming a Cournot model of competition, use a table similar to the one used in Example 4.8 to calculate the equilibrium point of this market (price, quantity, production, and profit of each firm). (Hint: Use a spreasheet. A resolution of 5 MW is acceptable)
\newline
First we have to find the Profit Maximimation of Firm A, or $\Omega_A$, is simply the revenue minus its cost. However, the price, $\pi$, depends on the output of both $P_A$ and $P_B$. Therefore we shall get the following equations:
$$ \Omega_A = \pi*P_A - C_A*P_A = (120 - (P_A + P_B))*P_A - 36*P_A = 120*P_A - P_A^2 - P_A*P_B - 36*P_A $$
$$ \Omega_A = 84*P_A - P_A^2 - P_A*P_B $$
Now, to maximize profit (and find Firm A's reaction curve), we simply find when the derivative (with respect to $P_A$) is zero:
$$ \frac{d\Omega_A}{dP_A} = 0 \rightarrow 84 - 2*P_A - P_B = 0 \rightarrow 2*P_A = 84 - P_B \rightarrow P_A = 42 - \frac{P_B}{2} $$
Now, to find Firm B's profit maximization equation, and respective reaction curve we go through the exact same steps. First we must find the $\Omega_B$:
$$ \Omega_B = \pi*P_B - C_B*P_B = (120 - (P_A + P_B))*P_B - 31*P_B = 120*P_B - P_A*P_B - P_B^2 - 31*P_B $$
$$ \Omega_B = 89*P_B - P_B^2 - P_A*P_B $$
And after, we now find the derivative (with respect to $P_B$) and set it to zero:
$$ \frac{d\Omega_B}{dP_B} = 0\rightarrow 0 = 89 - 2*P_B - P_A \rightarrow 2*P_B = 89 - P_A \rightarrow P_B = \frac{89-P_B}{2} $$
Now, to find equilibrium we simply take these two equations, and substitute one of them into the other. In this case I will be substituting the $P_B$ that we found above from Firm B into Firm A's reaction curve:
$$ P_A = 42 - \frac{\frac{89-P_A}{2}}{2} \rightarrow P_A = 42 - \frac{89 - P_A}{4} \rightarrow \frac{3}{4}*P_A = \frac{79}{4} \rightarrow P_A = \frac{79}{4}*\frac{4}{3} = \frac{79}{3} = 26.33\text{... MWh} \approx \textbf{25 MWh}$$
Now, to find $P_B$, we simply substitute the exact $P_A$ we found into Firm B's reaction curve:
$$ P_B = \frac{89 - \frac{79}{3}}{2} = \frac{\frac{188}{3}}{2} = \frac{188}{6} = \frac{94}{3} = 31.33\text{... MWh} \approx \textbf{30 MWh} $$
\begin{center}
  (The rounding/approximation was done due to the \textit{hint} stating that it is recommended).
\end{center}
\pagebreak
Now, to find the Equilibrium results we simply will be using the \textit{rounded} values of $P_A$ and $P_B$:
\begin{itemize}
\item Total Demand (D) $= P_A + P_B = 25+30 = \textbf{55}\text{ MW}$
  \item Market Price ($\pi$) $= 120 - D= 120 - 55 = \textbf{65}\frac{\$}{\text{MWh}}$
  \item Firm A Profit ($\Omega_A$) $= (\pi*P_A) - (C_A*P_A) = 65*25-36*25=1625-900=\$\textbf{725}$
  \item Firm B Profit ($\Omega_B$) $= (\pi*P_B) - (C_B*P_B) = 65*30-31*30=1950-930=\$\textbf{1020}$
\end{itemize}
And now, putting this all into a Cournot Iteration Table, it will be the following:
\begin{center}
  \begin{tabular}{|cccccc|}
    \hline
    \textbf{Step} & \textbf{Assumed $P_B$} & \textbf{Optimal $P_A$} & \textbf{Assumed $P_A$} & \textbf{Optimal $P_B$} & \textbf{Market Price $\pi$} \\
    \hline
    1 & 0 & 40 (42) & 40 & 25 (24.5) & 55 \\
    2 & 25 & 30 (29.5) & 30 & 30 (29.5) & 60 \\
    3 & 30 & \textbf{25} (27) & \textbf{25} & \textbf{30} (32) & \textbf{65} \\
    \hline
  \end{tabular}
\end{center}
Now, what in the world does the table mean? The table shows that if Firm B produces at the 30 MW, then Firm A's best response to that is to produce 25 MW. And if, say Firm A starts producing 25 MW, then Firm B's response should be \textit{logically} to produce 30 MW. Meaning, that it is a stable Nash Equilibrium.
\end{document}
