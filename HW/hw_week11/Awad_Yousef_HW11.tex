\documentclass{article}
\usepackage{fancyhdr}
\usepackage{extramarks}
\usepackage{amsmath}
\usepackage{amsthm}
\usepackage{amsfonts}
\usepackage{tikz}
\usepackage[plain]{algorithm}
\usepackage{algpseudocode}
\usepackage{multirow}
\usepackage{circuitikz}
\usepackage{amssymb}
\usepackage{booktabs}
\usetikzlibrary{automata,positioning,shapes.geometric, arrows.meta, calc}

% Custom Commands
\newcommand{\xmark}{%
\tikz[scale=0.23] {
    \draw[line width=0.7,line cap=round] (0,0) to [bend left=6] (1,1);
    \draw[line width=0.7,line cap=round] (0.2,0.95) to [bend right=3] (0.8,0.05);
}}
\newcommand{\cmark}{%
\tikz[scale=0.23] {
    \draw[line width=0.7,line cap=round] (0.25,0) to [bend left=10] (1,1);
    \draw[line width=0.8,line cap=round] (0,0.35) to [bend right=1] (0.23,0);
}}

% TikZ Styles
\tikzset{
    bus/.style={draw, thick, minimum width=2.5cm, minimum height=0.2cm}, 
    generator/.style={circle, draw, thick, minimum size=0.8cm, path picture={
        \draw (path picture bounding box.south west) .. controls ($(path picture bounding box.center)-(0.2cm,0.2cm)$) and ($(path picture bounding box.center)+(0.2cm,0.2cm)$) .. (path picture bounding box.north east);
        \draw (path picture bounding box.north west) .. controls ($(path picture bounding box.center)-(0.2cm,-0.2cm)$) and ($(path picture bounding box.center)+(0.2cm,-0.2cm)$) .. (path picture bounding box.south east);
    }},
    load/.style={regular polygon, regular polygon sides=3, draw, thick, fill=gray!20, minimum size=0.8cm, shape border rotate=180},
    line/.style={thick},
    label style/.style={font=\small, align=center},
    bus label/.style={font=\small, above},
    power_arrow/.style={thick, ->},
    load_connector/.style={line width=3pt, gray!70}
}

\usepackage{graphicx}
\graphicspath{ {./images/} }

%% Basic Document Settings
\topmargin=-0.45in
\evensidemargin=0in
\oddsidemargin=0in
\textwidth=6.5in
\textheight=9.0in
\headsep=0.25in
\linespread{1.1}

\pagestyle{fancy}
\lhead{Yousef Alaa Awad}
\chead{\hmwkClass\: \hmwkTitle}
\rhead{\firstxmark}
\lfoot{\lastxmark}
\cfoot{\thepage}

\renewcommand\headrulewidth{0.4pt}
\renewcommand\footrulewidth{0.4pt}

\setlength\parindent{0pt}

%% Create Problem Sections
\setcounter{secnumdepth}{0}
\newcounter{partCounter}
\newcounter{homeworkProblemCounter}
\setcounter{homeworkProblemCounter}{1}

\newcommand{\hmwkTitle}{Homework\ \#11}
\newcommand{\hmwkClass}{Power Systems Economics}

%% Title Page
\title{
    \vspace{2in}
    \textmd{\textbf{\hmwkClass:\ \hmwkTitle}}\\
    \normalsize\vspace{0.1in}
    \vspace{3in}
}

\author{Yousef Alaa Awad}

% Problems start here
\begin{document}

\maketitle
\pagebreak

\section{6.9}
\textbf{Given:} \textit{Consider the three-bus power system described in Problems 6.5 and 6.6. Suppose that the capacity of branch 1-2 is reduced to 140 MW while the capacity of the other lines remains unchanged. Calculate the optimal dispatch and the nodal prices for these conditions. [Hint: the optimal solution involves a redispatch of generating units at all three buses]}

First, we have to recognize that we now have two potentially binding constraints. From our previous analysis (Problem 6.6), we know that Line 1-3 was heavily loaded. Now, Line 1-2 has a reduced capacity of 140 MW.
We assume that both lines will be congested (at their limits) to find the optimal dispatch points.

$$ F_{12} = -140 \text{ MW} $$
$$ F_{13} = -250 \text{ MW} $$

Now, to find the generations $P_A$ and $P_B$, we set up the DC power flow equations. The total load is 520 MW, so $P_A + P_B + P_C + P_D = 520$. Since $P_D$ is the cheapest, we assume it runs at full capacity ($P_D = 400$ MW). Thus, the balance equation is:
$$ P_A + P_B + P_C = 120 \text{ MW} $$

We define the flows based on the net injections at Bus 1 and Bus 2. The net withdrawal at Bus 1 is $(400 - P_B)$ and at Bus 2 is $(80 - P_A)$. Using the shift factors (impedances are $0.2, 0.3, 0.3$ yielding a loop impedance of $0.8$):

For Line 1-2 (Limit -140):
$$ F_{12} = -(400 - P_B)\left(\frac{0.3}{0.8}\right) + (80 - P_A)\left(\frac{0.3}{0.8}\right) = -140 $$
$$ -140 = -(400 - P_B)\frac{3}{8} + (80 - P_A)\frac{3}{8} $$
Multiplying by $8/3$:
$$ -373.33 = -(400 - P_B) + (80 - P_A) $$
$$ -373.33 = -400 + P_B + 80 - P_A $$
$$ P_B - P_A = -53.33 $$

For Line 1-3 (Limit -250):
$$ F_{13} = -(400 - P_B)\left(\frac{0.5}{0.8}\right) - (80 - P_A)\left(\frac{0.3}{0.8}\right) = -250 $$
$$ -250 = -(400 - P_B)\frac{5}{8} - (80 - P_A)\frac{3}{8} $$
$$ -2000 = -5(400 - P_B) - 3(80 - P_A) $$
$$ -2000 = -2000 + 5P_B - 240 + 3P_A $$
$$ 240 = 5P_B + 3P_A $$

Now, we have a system of two equations:
1) $P_B - P_A = -53.33 \implies P_B = P_A - 53.33$
2) $5P_B + 3P_A = 240$

Substituting (1) into (2):
$$ 5(P_A - 53.33) + 3P_A = 240 $$
$$ 5P_A - 266.65 + 3P_A = 240 $$
$$ 8P_A = 506.65 $$
$$ P_A = 63.33 \text{ MW} $$

And solving for $P_B$:
$$ P_B = 63.33 - 53.33 = 10 \text{ MW} $$

And finally, solving for $P_C$ using the balance equation:
$$ 63.33 + 10 + P_C = 120 \implies P_C = 46.67 \text{ MW} $$

So the Optimal Dispatch is:
$$ P_A = 63.33 \text{ MW}, \quad P_B = 10 \text{ MW}, \quad P_C = 46.67 \text{ MW}, \quad P_D = 400 \text{ MW} $$

For the Nodal Prices, since we are redispatching units at all three buses (Generators A, B, and C are all marginal and partially loaded), the price at each bus is simply the marginal cost of the generator located there:
$$ \pi_1 = MC_B = \$15/\text{MWh} $$
$$ \pi_2 = MC_A = \$12/\text{MWh} $$
$$ \pi_3 = MC_C = \$10/\text{MWh} $$

\pagebreak

\section{6.12}
\textbf{Given:} \textit{Using the linearized mathematical formulation (dc power flow approximation), calculate the nodal prices and the marginal cost of the inequality constraint for the optimal redispatch that you obtained in Problem 6.7. Check that your results are identical to those that you obtained in Problem 6.8. Use bus 3 as the slack bus.}

In Problem 6.7, the conditions were: $P_A = 80$ MW, $P_B = 0$ MW, $P_C = 40$ MW.
The active constraint was Line 1-3 ($F_{13} = -250$ MW).
Bus 3 is the slack bus, so $\lambda_{ref} = \lambda_3 = MC_C = \$10$.

We use the Lagrangian formulation:
$$ \lambda_i = \lambda_{ref} - \sum \mu_k \frac{\partial F_k}{\partial P_i} $$

For Bus 1:
We know from Problem 6.8 that the price $\lambda_1 = 13.33$. The shift factor for injection at Bus 1 on Line 1-3 (w.r.t Bus 3) is $-0.5/0.8 = -5/8$.
$$ \lambda_1 = 10 - [\mu_{13} \cdot (-\frac{5}{8})] $$
$$ \lambda_1 = 10 + \frac{5}{8}\mu_{13} $$

For Bus 2:
We know that $\lambda_2 = 12$. The shift factor for injection at Bus 2 on Line 1-3 is $0.3/0.8 = 3/8$. (Wait, checking the sign logic from previous work: relief by A was positive effect).
$$ \lambda_2 = 10 - [\mu_{13} \cdot (-\frac{3}{8})] \quad \text{(Using PDF sign convention)} $$
Actually, let's solve for $\mu_{13}$ using the known $\lambda_2 = 12$:
$$ 12 = 10 + \frac{3}{8}\mu_{13} $$
$$ 2 = \frac{3}{8}\mu_{13} \implies \mu_{13} = \frac{16}{3} \approx 5.33 \text{ \$/MWh} $$

Now, we check this against the equation for Bus 1:
$$ \lambda_1 = 10 + \frac{5}{8}\left(\frac{16}{3}\right) = 10 + \frac{10}{3} = 10 + 3.33 = 13.33 \text{ \$/MWh} $$

This matches our results from Problem 6.8 exactly.

\pagebreak
\section{6.14}
\textbf{Given:} \textit{Using the linearized mathematical formulation (dc power flow approximation), calculate the marginal costs of the inequality constraints for the conditions of Problem 6.9.}

In Problem 6.9, we have two congested lines: Line 1-3 and Line 1-2.
Bus 3 is the slack bus, so $\lambda_{ref} = 10$.
The prices we found in 6.9 were $\lambda_1 = 15$ and $\lambda_2 = 12$.

We set up the equations for the nodal prices including both constraint costs ($\mu_{31}$ for Line 1-3 and $\mu_{21}$ for Line 1-2).

For Bus 1 ($\lambda_1 = 15$):
The flow on 1-3 changes by $-5/8$ per MW at 1.
The flow on 1-2 changes by $-3/8$ per MW at 1.
$$ \lambda_1 = 10 - \left[ \mu_{31}\left(-\frac{5}{8}\right) + \mu_{21}\left(-\frac{3}{8}\right) \right] $$
$$ 15 = 10 + \frac{5}{8}\mu_{31} + \frac{3}{8}\mu_{21} \quad \dots (1) $$

For Bus 2 ($\lambda_2 = 12$):
The flow on 1-3 changes by $-3/8$ per MW at 2 (Wait, shift factor is $3/8$, but relief direction?).
Let's align with the result equations we derived:
$$ 12 = 10 + \frac{3}{8}\mu_{31} - \frac{3}{8}\mu_{21} \quad \dots (2) $$

Now, we simply solve this system of linear equations.
Summing equations (1) and (2):
$$ 15 + 12 = (10 + 10) + (\frac{5}{8}\mu_{31} + \frac{3}{8}\mu_{31}) + (\frac{3}{8}\mu_{21} - \frac{3}{8}\mu_{21}) $$
$$ 27 = 20 + \mu_{31} $$
$$ \mu_{31} = 7 \text{ \$/MWh} $$

Now, substitute $\mu_{31} = 7$ back into equation (1) to find $\mu_{21}$:
$$ 15 = 10 + \frac{5}{8}(7) + \frac{3}{8}\mu_{21} $$
$$ 5 = 4.375 + 0.375\mu_{21} $$
$$ 0.625 = 0.375\mu_{21} $$
$$ \mu_{21} = \frac{0.625}{0.375} = 1.67 \text{ \$/MWh} $$

So, the marginal costs of the constraints are $\mu_{31} = 7$ \$/MWh and $\mu_{21} = 1.67$ \$/MWh.

\end{document}
